%%%%%%%%%%%%%%%%%%%%%%%%%%%%%%%%%%%%%%%%%%%%%%%%%%%%%%%%%%%%%%%%%%%%%%%%%%%%%%%%%%%%%%%
%%%%%%%%%%%%%%%%%%%%%%%%%%%%%%%%%%%%%%%%%%%%%%%%%%%%%%%%%%%%%%%%%%%%%%%%%%%%%%%%%%%%%%%
% 
% This top part of the document is called the 'preamble'.  Modify it with caution!
%
% The real document starts below where it says 'The main document starts here'.

\documentclass[12pt]{article}

\usepackage{amssymb,amsmath,amsthm}
\usepackage[top=1in, bottom=1in, left=2in, right=2in]{geometry}
\usepackage{fancyhdr}
\usepackage{graphicx}
\usepackage{enumerate}
\usepackage{verbatim}

% Comment the following line to use TeX's default font of Computer Modern.
\usepackage{times,txfonts}

\newtheoremstyle{homework}% name of the style to be used
  {18pt}% measure of space to leave above the theorem. E.g.: 3pt
  {12pt}% measure of space to leave below the theorem. E.g.: 3pt
  {}% name of font to use in the body of the theorem
  {}% measure of space to indent
  {\bfseries}% name of head font
  {:}% punctuation between head and body
  {2ex}% space after theorem head; " " = normal interword space
  {}% Manually specify head
\theoremstyle{homework} 

% Set up an Exercise environment and a Solution label.
\newtheorem*{exercisecore}{\@currentlabel}
\newenvironment{exercise}[1]
{\def\@currentlabel{#1}\exercisecore}
{\endexercisecore}

\newcommand{\localhead}[1]{\par\smallskip\noindent\textbf{#1}\nobreak\\}%
\newcommand\solution{\localhead{Solution:}}

% \newcommand{includematlab}[1]{\verbatiminput{#1}}

%%%%%%%%%%%%%%%%%%%%%%%%%%%%%%%%%%%%%%%%%%%%%%%%%%%%%%%%%%%%%%%%%%%%%%%%
%
% Stuff for getting the name/document date/title across the header
\makeatletter
\RequirePackage{fancyhdr}


% Shortcuts for blackboard bold number sets (reals, integers, etc.)
\newcommand{\Reals}{\ensuremath{\mathbb R}}
\newcommand{\Nats}{\ensuremath{\mathbb N}}
\newcommand{\Ints}{\ensuremath{\mathbb Z}}
\newcommand{\Rats}{\ensuremath{\mathbb Q}}
\newcommand{\Cplx}{\ensuremath{\mathbb C}}
%% Some equivalents that some people may prefer.
\let\RR\Reals
\let\NN\Nats
\let\II\Ints
\let\CC\Cplx

%%%%%%%%%%%%%%%%%%%%%%%%%%%%%%%%%%%%%%%%%%%%%%%%%%%%%%%%%%%%%%%%%%%%%%%%%%%%%%%%%%%%%%%
%%%%%%%%%%%%%%%%%%%%%%%%%%%%%%%%%%%%%%%%%%%%%%%%%%%%%%%%%%%%%%%%%%%%%%%%%%%%%%%%%%%%%%%
% 
% The main document start here.

% The following commands set up the material that appears in the header.
\title{Quiz 4}
\author{Andrew Player}
\date{\today}

\begin{document}

\maketitle

\begin{exercise}{Question \# 1}
\end{exercise}
\noindent
This Code of Conduct provides a good general guideline for how to act 
in any career, and especially in Computer related careers. To me, applying 
these rules means to generally be nice and respectful and to use common sense 
in the workplace. It would also mean continuing professional developement, keeping
up with new technologies, and striving to be a quality and professional worker.


\begin{exercise}{Question \# 2}
\end{exercise}
\noindent
Boyer says that safety-critical software should be defined in a more broad
sense than safety-critical systems. Safety-Critical Systems are defined to 
be those systems which could cause a loss of life if they failed. So, Boyer 
believes that safety-critical software should include all software which runs 
on systems, or for structures, whose failure could result in serious injuries
or loss of life. For example, software used to plan and test the design of 
bridges would be saftey-critical software.

\begin{exercise}{Question \# 3}
\end{exercise}
\noindent
I believe that Codes of Conduct can provide a standard for when to whistleblow, 
but not necessarily. For example, the ACM Code of Conduct specifically says
to report, to them, when a breach of the Code of Conduct is observed. However,
guidelines such as that one only provide guidance on reporting when the Code of 
Conduct itself is explicitly breached; therefore if some action is in a gray area,
there is not necessarily guidance on whether to report it or not. Also, they may not provide 
guidance on who to report to, or whether someone should inform the public if their 
report goes unnoticed. So, in my opinion, a Code of Conduct can provide good guidance on
when to whistleblow so long as it addresses those issues.
\newline
\newline
\newline
\newline
 
\begin{exercise}{Question \# 4}
\end{exercise}
\noindent
Typography matters because reader attention is a very limited resource. If you aren't 
able to properlly grab and maintain the readers attention, you will lose it. This can 
be very problematic if you are righting something that is important, especially if the 
writing has information that may be needed to keep people safe. Without proper Typography
you may also misguide the readers attention and cause them to skip over important details.
Furthermore, bad typography can be distracting and cause the reader to focus more on the act
of reading than on the message of your writing.

\end{document}