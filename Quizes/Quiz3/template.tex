%%%%%%%%%%%%%%%%%%%%%%%%%%%%%%%%%%%%%%%%%%%%%%%%%%%%%%%%%%%%%%%%%%%%%%%%%%%%%%%%%%%%%%%
%%%%%%%%%%%%%%%%%%%%%%%%%%%%%%%%%%%%%%%%%%%%%%%%%%%%%%%%%%%%%%%%%%%%%%%%%%%%%%%%%%%%%%%
% 
% This top part of the document is called the 'preamble'.  Modify it with caution!
%
% The real document starts below where it says 'The main document starts here'.

\documentclass[12pt]{article}

\usepackage{amssymb,amsmath,amsthm}
\usepackage[top=1in, bottom=1in, left=2in, right=2in]{geometry}
\usepackage{fancyhdr}
\usepackage{graphicx}
\usepackage{enumerate}
\usepackage{verbatim}

% Comment the following line to use TeX's default font of Computer Modern.
\usepackage{times,txfonts}

\newtheoremstyle{homework}% name of the style to be used
  {18pt}% measure of space to leave above the theorem. E.g.: 3pt
  {12pt}% measure of space to leave below the theorem. E.g.: 3pt
  {}% name of font to use in the body of the theorem
  {}% measure of space to indent
  {\bfseries}% name of head font
  {:}% punctuation between head and body
  {2ex}% space after theorem head; " " = normal interword space
  {}% Manually specify head
\theoremstyle{homework} 

% Set up an Exercise environment and a Solution label.
\newtheorem*{exercisecore}{\@currentlabel}
\newenvironment{exercise}[1]
{\def\@currentlabel{#1}\exercisecore}
{\endexercisecore}

\newcommand{\localhead}[1]{\par\smallskip\noindent\textbf{#1}\nobreak\\}%
\newcommand\solution{\localhead{Solution:}}

% \newcommand{includematlab}[1]{\verbatiminput{#1}}

%%%%%%%%%%%%%%%%%%%%%%%%%%%%%%%%%%%%%%%%%%%%%%%%%%%%%%%%%%%%%%%%%%%%%%%%
%
% Stuff for getting the name/document date/title across the header
\makeatletter
\RequirePackage{fancyhdr}


% Shortcuts for blackboard bold number sets (reals, integers, etc.)
\newcommand{\Reals}{\ensuremath{\mathbb R}}
\newcommand{\Nats}{\ensuremath{\mathbb N}}
\newcommand{\Ints}{\ensuremath{\mathbb Z}}
\newcommand{\Rats}{\ensuremath{\mathbb Q}}
\newcommand{\Cplx}{\ensuremath{\mathbb C}}
%% Some equivalents that some people may prefer.
\let\RR\Reals
\let\NN\Nats
\let\II\Ints
\let\CC\Cplx

%%%%%%%%%%%%%%%%%%%%%%%%%%%%%%%%%%%%%%%%%%%%%%%%%%%%%%%%%%%%%%%%%%%%%%%%%%%%%%%%%%%%%%%
%%%%%%%%%%%%%%%%%%%%%%%%%%%%%%%%%%%%%%%%%%%%%%%%%%%%%%%%%%%%%%%%%%%%%%%%%%%%%%%%%%%%%%%
% 
% The main document start here.

% The following commands set up the material that appears in the header.
\title{Quiz 3}
\author{Andrew Player}
\date{\today}

\begin{document}

\maketitle

\begin{exercise}{Question \# 1}
\end{exercise}
\noindent
A counterexample is something that is used to show that an arguement
is invalid. It is a case, which is logically possible, where the 
conclusion of the arguement is not garunteed to be true, while the 
assumptions are be true. Since you cannot gaurentee the conclusion 
to be true given true assumptions, the arguement is invalid.


\begin{exercise}{Question \# 2}
\end{exercise}
\noindent
An inductive argument may be invalid, as it does not gaurentee the 
truth of its conclusion, but instead it provides a high probability 
of truth. This is in contrast to a fallacious argument where the 
conclusion cannot be guarenteed to be true, nor is it likely to be 
true. Heres an example of an inductive argument: Every time I have
torrented something from Pirate Bay, nothing has happened; so next
time I torrent something from Pirate Bay, nothing will happen. 
Here's an example of a fallacious argument: My friend, who is a 
doctor, said that masks don't help to slow the spread of COVID, so 
masks don't help slow the spread of COVID.

\begin{exercise}{Question \# 3}
\end{exercise}
\noindent
The Appeal to Authority is a logical fallacy. It involves asserting 
that a claim is true because an authority on the topic said that it 
was true. It is a fallacy because they lack any supporting evidence 
other than the claim made by the authority. Here's an example:
John McAfee knows a lot about computers and cybersecurity, and 
he said Bitcoin will reach \$100,000 by the end of 2020, so that
must be true.
\newline
\newline
\newline
\newline
 
\begin{exercise}{Question \# 4}
\end{exercise}
\noindent
I read two sub-sections: ligatures and math symbols. This is what I
learned. Ligatures first came about since metal cast fonts would have
collisions between certain characters, so they simply made combined 
characters that fixed that issue. And now, we don't really need 
ligatures so they are largely a stylistic choice. Additionally, I 
learned that it is improper to use x for multiplication and that it 
is better to insert the proper characters for equations.

\end{document}