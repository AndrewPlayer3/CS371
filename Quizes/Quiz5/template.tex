%%%%%%%%%%%%%%%%%%%%%%%%%%%%%%%%%%%%%%%%%%%%%%%%%%%%%%%%%%%%%%%%%%%%%%%%%%%%%%%%%%%%%%%
%%%%%%%%%%%%%%%%%%%%%%%%%%%%%%%%%%%%%%%%%%%%%%%%%%%%%%%%%%%%%%%%%%%%%%%%%%%%%%%%%%%%%%%
% 
% This top part of the document is called the 'preamble'.  Modify it with caution!
%
% The real document starts below where it says 'The main document starts here'.

\documentclass[12pt]{article}

\usepackage{amssymb,amsmath,amsthm}
\usepackage[top=1in, bottom=1in, left=2in, right=2in]{geometry}
\usepackage{fancyhdr}
\usepackage{graphicx}
\usepackage{enumerate}
\usepackage{verbatim}

% Comment the following line to use TeX's default font of Computer Modern.
\usepackage{times,txfonts}

\newtheoremstyle{homework}% name of the style to be used
  {18pt}% measure of space to leave above the theorem. E.g.: 3pt
  {12pt}% measure of space to leave below the theorem. E.g.: 3pt
  {}% name of font to use in the body of the theorem
  {}% measure of space to indent
  {\bfseries}% name of head font
  {:}% punctuation between head and body
  {2ex}% space after theorem head; " " = normal interword space
  {}% Manually specify head
\theoremstyle{homework} 

% Set up an Exercise environment and a Solution label.
\newtheorem*{exercisecore}{\@currentlabel}
\newenvironment{exercise}[1]
{\def\@currentlabel{#1}\exercisecore}
{\endexercisecore}

\newcommand{\localhead}[1]{\par\smallskip\noindent\textbf{#1}\nobreak\\}%
\newcommand\solution{\localhead{Solution:}}

% \newcommand{includematlab}[1]{\verbatiminput{#1}}

%%%%%%%%%%%%%%%%%%%%%%%%%%%%%%%%%%%%%%%%%%%%%%%%%%%%%%%%%%%%%%%%%%%%%%%%
%
% Stuff for getting the name/document date/title across the header
\makeatletter
\RequirePackage{fancyhdr}


% Shortcuts for blackboard bold number sets (reals, integers, etc.)
\newcommand{\Reals}{\ensuremath{\mathbb R}}
\newcommand{\Nats}{\ensuremath{\mathbb N}}
\newcommand{\Ints}{\ensuremath{\mathbb Z}}
\newcommand{\Rats}{\ensuremath{\mathbb Q}}
\newcommand{\Cplx}{\ensuremath{\mathbb C}}
%% Some equivalents that some people may prefer.
\let\RR\Reals
\let\NN\Nats
\let\II\Ints
\let\CC\Cplx

%%%%%%%%%%%%%%%%%%%%%%%%%%%%%%%%%%%%%%%%%%%%%%%%%%%%%%%%%%%%%%%%%%%%%%%%%%%%%%%%%%%%%%%
%%%%%%%%%%%%%%%%%%%%%%%%%%%%%%%%%%%%%%%%%%%%%%%%%%%%%%%%%%%%%%%%%%%%%%%%%%%%%%%%%%%%%%%
% 
% The main document start here.

% The following commands set up the material that appears in the header.
\title{Quiz 5}
\author{Andrew Player}
\date{\today}

\begin{document}

\maketitle

\begin{exercise}{Question \# 1}
\end{exercise}
\noindent
For the most part, I was taught these ethics growing up. The only parts that 
I was not taught are the computing specific ones. For example, I was not taught
to acknowledge that all people are stakeholders in computing. I was also not 
explicitly told that I should contribute to society and human well being. 


\begin{exercise}{Question \# 2}
\end{exercise}
\noindent
"Privacy in Public" refers to the problem of protecting public personal information.
For the most part, there is no legal protection for what is considered public personal
information, even though some people would not want that information just given away. For
example, the things that you purchase in a store, or online, are considered PPI, but you 
may not want people to see what you are buying. There are many other situations where you
would want "public" information kept relatively private, but since there are no protections
for this information, it may be passed around without your explicit permission. That is the problem.

\begin{exercise}{Question \# 3}
\end{exercise}
\noindent
The US government uses several controversial survailence techniques. For example, they have means to
intercept and read emails of US citizens. They also have to ability to get the searches 
made by US citizens on certain search engines. These are considered controversial as there is almost
zero accountability for the governments use and gathering of peoples' data. Many people 
believe that the government shouldn't have access to any "online" data without having a warrant, 
like if the data was on paper, as it could be considered a violation of the 4th Amendment. 
\newline 

\begin{exercise}{Question \# 4}
\end{exercise}
\noindent
James Moor connects these by saying that an individual only has privacy if 
they are protect from intrusion, interference, and information access by others 
in that situation. Moor means "situation" in a very broad sense, where it refers
to any context where the idea of privacy can be applied. He then further differentiates
situations into naturally private situations and normatively private situations.
Naturally private situations are those where individuals are protected by natural things, 
such as being alone in the woods. Normatively private situations are those where
people are protected by "norms" such as laws, policies, and other social constructs.

\end{document}