%%%%%%%%%%%%%%%%%%%%%%%%%%%%%%%%%%%%%%%%%%%%%%%%%%%%%%%%%%%%%%%%%%%%%%%%%%%%%%%%%%%%%%%
%%%%%%%%%%%%%%%%%%%%%%%%%%%%%%%%%%%%%%%%%%%%%%%%%%%%%%%%%%%%%%%%%%%%%%%%%%%%%%%%%%%%%%%
% 
% This top part of the document is called the 'preamble'.  Modify it with caution!
%
% The real document starts below where it says 'The main document starts here'.

\documentclass[12pt]{article}

\usepackage{amssymb,amsmath,amsthm}
\usepackage[top=1in, bottom=1in, left=2in, right=2in]{geometry}
\usepackage{fancyhdr}
\usepackage{graphicx}
\usepackage{enumerate}
\usepackage{verbatim}

% Comment the following line to use TeX's default font of Computer Modern.
\usepackage{times,txfonts}

\newtheoremstyle{homework}% name of the style to be used
  {18pt}% measure of space to leave above the theorem. E.g.: 3pt
  {12pt}% measure of space to leave below the theorem. E.g.: 3pt
  {}% name of font to use in the body of the theorem
  {}% measure of space to indent
  {\bfseries}% name of head font
  {:}% punctuation between head and body
  {2ex}% space after theorem head; " " = normal interword space
  {}% Manually specify head
\theoremstyle{homework} 

% Set up an Exercise environment and a Solution label.
\newtheorem*{exercisecore}{\@currentlabel}
\newenvironment{exercise}[1]
{\def\@currentlabel{#1}\exercisecore}
{\endexercisecore}

\newcommand{\localhead}[1]{\par\smallskip\noindent\textbf{#1}\nobreak\\}%
\newcommand\solution{\localhead{Solution:}}

% \newcommand{includematlab}[1]{\verbatiminput{#1}}

%%%%%%%%%%%%%%%%%%%%%%%%%%%%%%%%%%%%%%%%%%%%%%%%%%%%%%%%%%%%%%%%%%%%%%%%
%
% Stuff for getting the name/document date/title across the header
\makeatletter
\RequirePackage{fancyhdr}


% Shortcuts for blackboard bold number sets (reals, integers, etc.)
\newcommand{\Reals}{\ensuremath{\mathbb R}}
\newcommand{\Nats}{\ensuremath{\mathbb N}}
\newcommand{\Ints}{\ensuremath{\mathbb Z}}
\newcommand{\Rats}{\ensuremath{\mathbb Q}}
\newcommand{\Cplx}{\ensuremath{\mathbb C}}
%% Some equivalents that some people may prefer.
\let\RR\Reals
\let\NN\Nats
\let\II\Ints
\let\CC\Cplx

%%%%%%%%%%%%%%%%%%%%%%%%%%%%%%%%%%%%%%%%%%%%%%%%%%%%%%%%%%%%%%%%%%%%%%%%%%%%%%%%%%%%%%%
%%%%%%%%%%%%%%%%%%%%%%%%%%%%%%%%%%%%%%%%%%%%%%%%%%%%%%%%%%%%%%%%%%%%%%%%%%%%%%%%%%%%%%%
% 
% The main document start here.

% The following commands set up the material that appears in the header.
\title{Quiz 6}
\author{Andrew Player}
\date{\today}

\begin{document}

\maketitle

\begin{exercise}{Question \# 1}
\end{exercise}
\noindent
While it does not make much sense to apply this to the game I am making for this class, I am
making a steganography application for my security class project which is easier to apply this to.
I could add a tutorial and learning section to the application that teaches people about what it does
and how it does it, along with information on stenography and cryptography, in general, that could
help some people learn about cybersecurity. If the application was going out to the public, it could
definitely help foster learning for its users.


\begin{exercise}{Question \# 2}
\end{exercise}
\noindent
I do believe that it is possible to make a hacker's code of ethics that does not engage in 
moral relativism. Moral Relativism states that "moral judgments are true or false only relative
to some particular standpoint." And it is possible to establish ethics that are not necessarily
specific to the technological world. For example, a rule such as "Never instigate" would apply well 
to the hacker world, while it is also an ethic that is not relative to the hacker world and can be 
applied to anything.

\begin{exercise}{Question \# 3}
\end{exercise}
\noindent
Cyberterrorism is the act of commiting terrorism within "cyberspace", i.e. over the internet
or on local computer networks. Terrorism in this sense is some action meant to harm or inspire
fear again some group of people, such as a company or government, for some type of gain (usually political). 
One example of this would be to hack into some critical infrastructure, such as a powerplant, and take 
control of it. 
\newline 
 

\newpage
\begin{exercise}{Question \# 4}
\end{exercise}
\noindent
Cybersecurity issues and technological privacy issues connected in many ways. The more data
companies collect, the more data that hackers could steal. The less privacy we have on the internet
the more vulnerable we are to lose important information to hackers. Thus the two ideas are directly linked.

\end{document}